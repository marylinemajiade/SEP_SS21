\chapter{Projektbeschränkungen}

\section{Beschränkungen}

\newcounter{lb}\setcounter{lb}{10}

\begin{description}[leftmargin=5em, style=sameline]
	
	\begin{lhp}{lb}{LB}{beschr:lehrbots}
		\item [Name:] Selbstlehrende Bots
		\item [Beschreibung:] Keine Selbstlehrfunktion von Bots wird implementiert.
		\item [Motivation:] Die Funktionalität ist zu aufwändig zu implementieren und passt deshalb nicht in das Zeitbudget.
		\item [Erfüllungskriterium:] Intelligenzalgorithmus von Bots ist so vorprogrammiert, dass sie Entscheidungen nur anhand des vorprogrammierten Wissens sowie des aktuellen Spielstands treffen, ohne dabei frühere Spiele zu berücksichtigen.
	\end{lhp}
	
	\begin{lhp}{lb}{LB}{beschr:anwendungsbereich}
		\item [Name:] Anwendungsbereich
		\item [Beschreibung:] Das System ist ausschließlich für den privaten Bereich ausgelegt.
		\item [Motivation:] Das SEP besitzt keine Lizenzen und Rechte an dem Spiel und verfolgt keine kommerziellen Ziele.
		\item [Erfüllungskriterium:] Die Software wird nur den Teilnehmern und Betreuern des SEP zugänglich gemacht.
	\end{lhp}
	
		
	\begin{lhp}{lb}{LB}{beschr:implsprache}
		\item [Name:] Implementierungssprache
		\item [Beschreibung:] Für die Implementierung ist ausschließlich Java 8 oder höher zu verwenden.
		\item [Motivation:] Das optimiert die Betreuung vom SEP und koordiniert die Mitarbeit.
		\item [Erfüllungskriterium:] Die Teilnehmer verpflichten sich zur ausschließlichen Verwendung von Java 8 oder höher und installieren entsprechende Versionen. Abgaben, die in einer älteren Version erstellt wurden, werden vom Betreuer nicht akzeptiert.
	\end{lhp}
	
	\begin{lhp}{lb}{LB}{beschr:gui}
		\item [Name:] GUI-Framework
		\item [Beschreibung:] Die GUI ist mit JavaFX zu realisieren.
		\item [Motivation:] Das optimiert die Betreuung vom SEP und koordiniert die Mitarbeit.
		\item [Erfüllungskriterium:] Projekte, die mit einem anderen GUI-Framework erstellt wurden, werden vom Betreuer abgelehnt und gelten als nicht bestanden.
	\end{lhp}
	
	\begin{lhp}{lb}{LB}{beschr:gitlab}
		\item [Name:] Gitlab
		\item [Beschreibung:] Für die Entwicklung ist das vorgegebene GitLab-Repository zu verwenden.
		\item [Motivation:] Das optimiert die Betreuung vom SEP und koordiniert die Mitarbeit.
		\item [Erfüllungskriterium:] Projekte, die in einem anderen Repository oder auf einer anderen Plattform verwaltet wurden, werden vom Betreuer nicht akzeptiert und gelten als nicht bestanden.
	\end{lhp}
	
	
\end{description}

\section{Glossar}

\begin{center}
		\rowcolors{2}{Gray!15}{White}
		\begin{longtable}{p{0.25\textwidth} p{0.25\textwidth} p{0.4\textwidth}}
			\textbf{Deutsch} & \textbf{Englisch} & \textbf{Bedeutung} \\
			\hline \hline \endhead
			Ablagestapel & discard pile & Stapel auf den die Karten im Spielverlauf nach den geltenden Regeln abgelegt werden. Die oberste Karte ist aufgedeckt\\
			Anwendung & application & Computerprogramm, hier zum Spielen von LAMA\\
			Aussteigen & (to) fold & Der Spieler kann keinen gültigen Zug mehr durchführen und muss bis zum Ende des Durchgangs warten\\
			Beispiele & Examples & Beispiele aus dem SEP letzter Jahren, welche angepasst werden müssen.\\ 
			Benutzer & user & Inhaber des Benutzerkontos und Spieler\\
			Benutzerkonto & account & Zugangsberechtigung zum Spiel\\
			Benutzername & nickname & Frei wählbarer, virtueller Name des Spielers, der dem Benutzerkonto zugeordnet ist\\
			Bestenliste & highscore & Liste der Benutzernamen mit den besten Wertungen\\
			Bot & bot & Spieler, dessen Spielaktionen vom Computer entschieden und durchgeführt werden\\
			Dienstleister & server & Programm, das auf Anfrage Zugang zu einem Dienst verschafft \\
			Durchgang & round & Zeitraum von Beginn einer Runde bis zu dem Moment, in dem alle Spieler ausgestiegen sind oder einer keine Karten mehr hat.
			Ein Spiel besteht aus mehreren Durchgängen.\\
			Handkarten & cards & (hier virtueller) Gegenstand des Spiels\\
			Kekse & Cookies & Offiziell keine gültige Maßnahme zur Bestechung der HiWis\\ 
			Kunde & client & Programm, das den Dienst des Servers anfordert\\
 			Lobby & lobby & Virtueller Raum zum Betreten eines Spielraums\\	
 			Mehrspieler & multiplayer & Das Spiel wird mit oder gegen andere Spieler gespielt.\\
 			Minuspunkte & negative points & Teil des Punktesystems, der über den Gewinner entscheidet\\
 			Nachziehstapel & draw pile & Übrige Karten, die nicht auf die Spieler aufgeteilt wurden.\\
 			Punktemarken & chips & Repräsentieren die Punkte im Spiel.\\
			Spiel (Regelwerk) & game & LAMA \\
			Spieler & player & Teilnehmer am Spielgeschehen\\
			Spielraum & game room & Virtueller Raum, in dem ein Spiel stattfindet\\
			Unterhaltungsecke & chatroom & Virtueller Raum, in dem man mit anderen Spielern kommunizieren kann.\\
			Wert & value & Eigenschaft von Handkarten, die eine jeweils feste Punkteanzahl repräsentiert.\\
			Zug & turn & Zustand in dem ein Spieler eine Spielaktion ausführen muss\\
		\end{longtable}
\end{center}

\section{Relevante Fakten und Annahmen}

Wichtige bekannte Fakten und getroffene Annahmen, die sich auf das Projekt direkt oder indirekt beziehen und dadruch auf die zukünftige Implementierungsentscheidungen Effekt haben können.

\newcounter{fa}\setcounter{fa}{10}

\begin{description}[leftmargin=5em, style=sameline]
	
	\begin{lhp}{fa}{FA}{fa:fortentwicklung}
		\item [Name:] Keine Fortentwicklung der App nach dem SEP.
		\item [Beschreibung:] Nach Ende des SEP wird das Projekt nicht weiterentwickelt.
		\item [Motivation:] Das Entwicklungsteam hat keine Lust darauf.
	\end{lhp}
	
	\begin{lhp}{fa}{FA}{fa:recht}
		\item [Name:] Keine Lizenzen für Spielartefakte.
		\item [Beschreibung:] Weder die TU Kaiserslautern noch das Spielwerk + die Freizeit GmbH gewahren dem Entwicklungsteam die Rechte für die Spielartefakte.
		\item [Motivation:] Rechtliche Vorsorge.
	\end{lhp}
	
	\begin{lhp}{fa}{FA}{fa:recht-vergangenheit}
		\item [Name:] Keine bekannte Nachteile von Verwendung von Spielartefakten.
		\item [Beschreibung:] Es ist nicht bekannt, dass die SEP-Teilnehmer der letzten Jahre irgendwelche rechtlichen Probleme dadurch gehabt haben, dass sie die Spielartefakten vom Spielwerk + der Freizeit GmbH im Rahmen des SEP eingesetzt haben.
		\item [Motivation:] Rechtliche Vorsorge.
	\end{lhp}
	
	
\end{description}

